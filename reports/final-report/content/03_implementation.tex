\section*{Implementation}

\subsection*{Development Environment} 
% TODO: add links
The development of Cuteserver was primarily accomplished through pair programming. We held regular meetings where we shared our terminals using a shell sharing tool called sshx. This tool was particularly effective for collaboration since we both used Neovim as our code editor. \\

Initially, we used Makefiles to build the source code, but this approach was inefficient due to frequent adjustments. As the project grew more complex, we decided to use CMake for more efficient project building and file generation. \\

Our implementation strategy for the web server was iterative, starting with small, manageable tasks and progressively tackling more complex ones. We regularly tested the code by running it with different inputs to ensure functionality.


\subsection*{Project Structure} % (also dasmer de tree zeige und kurz erkläre welles file für was isch)
% main (socket connection, thread pool, keep-alive) TODO: timeout implementieren  
% parse headers 

\subsection*{Static File Requests}
%error codes chätzli 

\subsection*{CGI}
% pipes 
% env variables 
% process creation (exec)

\subsection*{Server Configuration}
% Our goal was to provide a web server that needs minimal configuration. Therefore we put everything into 1 config file. 
% TOML File 
% Dockerfile  (readme für apps erstellen (dockerfiles))
From the beginning, we aimed for our web server to support hosting multiple resources connected to different domain names. To achieve this multidomain support, we needed to make the web server configurable. To ensure user-friendly configuration, we used TOML files. The web server parses a configuration file listing all resources and their locations for the server to serve. Additionally, users can specify resource-independent settings, such as the number of active worker threads and log storage locations. We also provide a command line interface where users can specify the TCP address and the path to the configuration file.


\subsection*{Example Application}
% TODO: synchronization 

We had several web applications to test our web server. Initially, we used simple HTML files and gradually added more complex ones. Eventually, we developed a more sophisticated application using the React framework for the front end. This application is a simple chat app that makes POST requests when users submit chat messages. These POST requests are handled by our backend service, which stores the chat messages in a JSON file and sends the JSON data to clients. We initially wrote this service using the Flask framework in Python to test if interpreted languages could work as CGI scripts. While this worked, we later translated the backend Python script to a C script to achieve faster response times by running compiled executables.

\subsection*{Containerization}

We decided to containerize our application for two main reasons. Initially, we used chroot jails to sandbox our application, i.e., a mechanism that isolates a process and its children from the rest of the system by changing their apparent root directory to a specified path. This approach worked until we implemented CGI support, which required the web server to start new processes with the necessary libraries for the CGI scripts. Manually copying those dependencies into the chroot jail would have been too laborious. Docker containers offered an easier solution by securely sandboxing execution and including dependencies. Additionally, Dockerfiles enabled us to automate the building of the web server and the example application, making it easy to use and run. We used multistage builds, first creating containers for building and then copying only the necessary files to the final containers. Multistage build reduced the container size from 1.2 GB to approximately 200 MB. This process enhanced our understanding of containerization.
